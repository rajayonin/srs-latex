%!TEX program = lualatex
\documentclass[a4paper,10pt]{article}

\usepackage[en,enableTraceability,enableCaptions]{SRS}  % requirements
\usepackage{hyperref}
\usepackage{placeins}  % \FloatBarrier


\begin{document}
  \section{User requirements}

  Template in Figure \ref{tab:ur-template}.

  \printureqtemplate

  % pc: Prioridad del cliente
  %  - h: high, m: medium, l: low
  % pd: Prioridad del desarrollador
  %  - h: high, m: medium, l: low
  % s: Estabilidad
  %  - nc: no cambia, c: cambia, vi: muy inestable
  % v: Verificabilidad
  %  - h: high, m: medium, l: low
  \begin{userReq}{CA}{name}{pc=m,pd=l,s=nc,v=m}
    Mi carro me lo robaron estando de romería.
  \end{userReq}

  \begin{userReq}{CA}{id}{pc=m,pd=l,s=nc,v=h}
    Mi carro me lo robaron anoche cuando dormía.
  \end{userReq}

  \begin{userReq}{RE}{name}{pc=m,pd=l,s=nc,v=m}
    Mi carro me lo robaron estando de romería.
  \end{userReq}

  \begin{userReq}{RE}{id}{pc=m,pd=l,s=nc,v=h}
    Mi carro me lo robaron anoche cuando dormía.
  \end{userReq}

  \FloatBarrier

  \ureqref{CA-name} in Figure \ref{req:UR-01}.

  \section{System requirements}
  Template in Figure \ref{tab:sr-template}.

  \printsreqtemplate

  \begin{softwareReq}{FN}{name}{pc=m,pd=l,s=nc,v=h}{CA-name,CA-id}
    ¿Dónde estará mi carro?
  \end{softwareReq}

  \begin{softwareReq}{FN}{id}{pc=m,pd=l,s=nc,v=h}{CA-id,CA-name}
    ¿Dónde estará mi carro?
  \end{softwareReq}

  \begin{softwareReq}{NF}{name}{pc=m,pd=l,s=nc,v=h}{RE-name}
    ¿Dónde estará mi carro?
  \end{softwareReq}

  \begin{softwareReq}{NF}{id}{pc=m,pd=l,s=nc,v=h}{RE-id}
    ¿Dónde estará mi carro?
  \end{softwareReq}

  \FloatBarrier


  \section{Use cases}
  Template in Table \ref{tab:uc-template}.

  \printuctemplate

  \begin{useCase}{id}
    {Recuperar el carro.}  % name
    {Manolo}  % actors
    {Quiere recuperar su carro.}  % objetivo
    {Se lo han robao.}  % pre-cond
    {Encarcelan al que lo ha hecho.}  % post-cond
    \begin{enumerate}  % description
      \item Lo busca
      \item Lo encuentra
      \item Profit
    \end{enumerate}
  \end{useCase}


  \FloatBarrier

  \ucref{id} in Table \ref{uc:UC-01}.

  \section{Requirement traceability}
  \begin{figure}[h]
    % Autogenerate traceability matrix for requirements that match the given Lua
    % patterns. Note that the `-' is a special character an should be escaped, using \37.
    \centering\traceabilityPrintMatrix{^SR}{^UR}{}
    \caption{Autogenerated traceability matrix}
  \end{figure}


  \FloatBarrier


  \section{Components}
  Template in Table \ref{tab:uc-template}.

  \printcomptemplate

  \begin{component}{Carro}
    {Ser robado}  % role
    {\NA}  % dependencias
    {gente}  % in-data
    {más gente}  % out-data
    {SR-01}  % origen
    % descripción
    Es mu bonito, pero propenso a ser robado.
  \end{component}

  \section{Traceability shorthands}
  \begin{figure}[h]
    \centering\traceabilityFNCA
    \caption{Autogenerated traceability matrix, SR-FN to UR-CA}
  \end{figure}
  \begin{figure}[h]
    \centering\traceabilityNFRE
    \caption{Autogenerated traceability matrix, SR-NF to UR-RE}
  \end{figure}
  \begin{figure}[h]
    \centering\traceabilityCompFN
    \caption{Autogenerated traceability matrix, components to SR-FN}
  \end{figure}
  \begin{figure}[h]
    \centering\traceabilityVETSR
    \caption{Autogenerated traceability matrix, VET tests to SR}
  \end{figure}
  \begin{figure}[h]
    \centering\traceabilityVATUR
    \caption{Autogenerated traceability matrix, VAT tests to UR}
  \end{figure}

\end{document}

